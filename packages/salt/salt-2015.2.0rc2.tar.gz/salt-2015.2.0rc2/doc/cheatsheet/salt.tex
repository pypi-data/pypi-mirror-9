\documentclass[landscape,a4paper,10pt]{article}
\usepackage{multicol}
\usepackage[landscape]{geometry}

\usepackage{concmath}
\usepackage[T1]{fontenc}

\geometry{top=.5in,left=.5in,right=.5in,bottom=.5in}

\pagestyle{empty}

\setcounter{secnumdepth}{0}

\setlength\columnseprule{.4pt}
\setlength{\parindent}{0pt}
\setlength{\parskip}{0pt plus 0.5ex}

\begin{document}

\begin{multicols}{3}

\bfseries{\Huge{Salt States}}\\[-0.3cm]
\line(1,0){125}\\

\bfseries{\LARGE{top.sls}}

The \texttt{top.sls} file is used to map what SLS modules get loaded onto what minions via the state system.\\

Here is an example \texttt{top.sls} file which uses \texttt{pkg}, \texttt{file} and \texttt{service} states:

\begin{verbatim}
apache:
  pkg:
    - installed
    - version: 2.2.23
  service:
    - running
    - enable: True
/var/www/index.html: # ID declaration
  file:              # state declaration
    - managed        # function
    - source: salt://webserver/index.html
    - user: root
    - group: root
    - mode: 644
    - require:       # requisite declaration
      - pkg: apache  # requisite reference
\end{verbatim}


\bfseries{\LARGE{file}}

Manages files.

FUNCTIONS


\begin{itemize}
  \item \texttt{absent}  -- verify that a file or directory is absent
  \item \texttt{accumulated}  -- prepare accumulator which can be used for template in \texttt{file.managed}
  \item \texttt{append}  -- put some text at the end of file
  \item \texttt{comment}  -- comment out some lines
  \item \texttt{directory}  -- ensure that a directory is present
  \item \texttt{exists}  -- ensure that a directory or file is present
  \item \texttt{managed}  -- this file is managed by the salt master and can be run through templating system
  \item \texttt{patch}  -- apply a patch to a file
  \item \texttt{recurse}  -- recurse through a subdirectory on master
  \item \texttt{rename}  -- rename a file
  \item \texttt{sed}  -- edit a file
  \item \texttt{symlink}  -- create a symlink
  \item \texttt{touch}  -- create an empty file or update the access and modification time
  \item \texttt{uncomment}  -- uncomment lines in a file
\end{itemize}

\begin{verbatim}
/etc/http/conf/http.conf:
  file.managed:
    - source: salt://apache/http.conf
    - user: root
    - group: root
    - mode: 644
    - template: jinja
    - context:
        custom_var: "override"
    - defaults:
        custom_var: "default value"
        other_var: 123
\end{verbatim}

\bfseries{\LARGE{pkg}}

Manage software packages.

FUNCTIONS

\begin{itemize}
  \item \texttt{installed} -- verify if a package is installed
  \item \texttt{latest} -- verify that the package is the latest version
  \item \texttt{purged}
  \item \texttt{removed} -- verify if the package is removed
\end{itemize}

\begin{verbatim}
httpd:
  pkg:
    - installed
    - repo: mycustomrepo
    - skip_verify: True
    - version: 2.0.6~ubuntu3
\end{verbatim}

\bfseries{\LARGE{service}}

Manage system daemons/services.

FUNCTIONS

\begin{itemize}
  \item \texttt{dead}
  \item \texttt{disabled}
  \item \texttt{enabled}
  \item \texttt{mod\_watch}
  \item \texttt{running}
\end{itemize}

\begin{verbatim}
apache:
  service:
    - running
    - name: httpd
    - enable: True
    - sig: httpd
\end{verbatim}


\bfseries{\LARGE{cmd}}

Execution of arbitrary commands.

FUNCTIONS

\begin{itemize}
  \item \texttt{mod\_watch}
  \item \texttt{run}
  \item \texttt{script}
  \item \texttt{wait}
  \item \texttt{wait\_script}
\end{itemize}

\begin{verbatim}
date > /tmp/salt-run:
  cmd:
    - run
\end{verbatim}

\bfseries{\LARGE{cron}}

Manage cron jobs (scheduler).

\begin{verbatim}
date > /tmp/crontest:
  cron.present:
    - user: root
    - minute: 7
    - hour: 2
\end{verbatim}

FUNCTIONS

\begin{itemize}
 \item \texttt{absent}
 \item \texttt{file} -- provides file.managed-like functionality (templating, etc.) for a pre-made crontab file, to be assigned to a given user
 \item \texttt{present}
\end{itemize}

\bfseries{\LARGE{user}}

The user module is used to create and manage user settings, users can be set as either absent or present.

\begin{verbatim}
fred:
  user.present:
    - fullname: Fred Jones
    - shell: /bin/zsh
    - home: /home/fred
    - uid: 4000
    - gid: 4000
    - groups:
      - wheel
      - storage
      - games

testuser:
  user.absent
\end{verbatim}

FUNCTIONS

\bfseries{\LARGE{group}}

The group module is used to create and manage unix group settings, groups can be either present or absent.

FUNCTIONS

\begin{itemize}
 \item \texttt{present}
 \item \texttt{absent}
\end{itemize}

\begin{verbatim}
cheese:
  group.present:
    - gid: 7648
    - system: True
\end{verbatim}

\bfseries{\LARGE{git}}

Interaction with Git repositories.

\begin{verbatim}
https://github.com/saltstack/salt.git:
  git.latest:
    - rev: develop
    - target: /tmp/salt
\end{verbatim}

\bfseries{\LARGE{host}}

Management of addresses and names in hosts file.

\begin{verbatim}
salt-master:
  host.present:
    - ip: 192.168.0.42
\end{verbatim}

\bfseries{\LARGE{kmod}}

Loading and unloading of kernel modules.

\begin{verbatim}
kvm_amd:
  kmod.present
pcspkr:
  kmod.absent
\end{verbatim}

\bfseries{\LARGE{mount}}

Mounting of filesystems.

\begin{verbatim}
/mnt/sdb:
  mount.mounted:
    - device: /dev/sdb1
    - fstype: ext4
    - mkmnt: True
    - opts:
      - defaults
\end{verbatim}

\bfseries{\LARGE{sysctl}}

Configuration of the Linux kernel using sysctl.

\begin{verbatim}
vm.swappiness:
  sysctl.present:
    - value: 20
\end{verbatim}

\end{multicols}
\end{document}

