\section{\module{CookieAuthenticate} --- Cookie-based authentication.}
\declaremodule{standard}{CookieAuthenticate}

\subsection{Module}
A cookie-based authentication allows website users to login/logout using a username and a password.

While they are logged in, their session information is stored on their computer via a cookie.

If they are inactive for too long, they are automatically logged out.

This module provides an easy to use implementation of a cookie-based authentication.

Unlike many cookie-based authentication methods, it doesn't require any database on the server side to
store session informations. It uses three cookies to store the session information:
\begin{itemize}
\item
One cookie called \var{CherryLogin} that contains the login of the user
\item
One cookie called \var{CherryDate} that contains the time of the last action
\item
One cookie called \var{CherryPassword} that contains the password of the user, encrypted with the login and the
time of the last action. This is to prevent someone from manually changing the last action time.
\end{itemize}

To use this module, you have to declare a CherryClass that inherits from \var{CookieAuthenticate}, and all your masks
and views will be automatically protected.

To perform this magic, \var{CookieAuthenticate} uses AOP (aspect oriented programming). This basically means
that it will add some extra code at the beginning of each of your masks and views.

You may use the following variables and methods:

\begin{memberdesc}{variable: loginCookieName}
String containing the name of the cookie where the \var{login} is stored. (default value is \var{CherryLogin})
\end{memberdesc}
\begin{memberdesc}{variable: dateCookieName}
String containing the name of the cookie where the \var{last action time} is stored. (default value is \var{CherryDate})
\end{memberdesc}
\begin{memberdesc}{variable: passwordCookieName}
String containing the name of the cookie where the \var{password} is stored. (default value is \var{CherryPassword})
\end{memberdesc}
\begin{memberdesc}{variable: timeout}
Integers containing the timeout in minutes. If the user is inactive for that time, it will
automatically be logged out. Default value ie 60. Set it to 0 if you want no timeout.
\end{memberdesc}

\begin{funcdesc}{function: getPasswordListForLogin}{login}
This is where you specify what the valid login/password combinations are. The input value is the login that the
user entered. The method should return a list of all valid passwords for this login. If the login is incorrect, just
return an empty list.

Note: Being able to return several matching passwords for a login allows you to keep a "master key" password that works 
with all logins.
\end{funcdesc}

\begin{funcdesc}{mask or view: loginScreen}{message, fromPage, login=''}
This is the page that is displayed when the user tries to access a protected page without being logged in.

\var{message} is a string containing the reason why no user is logged in. Possible values are:
\begin{itemize}
\item
\strong{timeoutMessage}: This means that someone was logged in, but they remained inactive for too long
\item
\strong{wrongLoginPasswordMessage}: This means that someone is trying to log in, but the login and password they
entered are incorrect
\item
\strong{noCookieMessage}: This means that no informations are available: this is probably the first time the
user is coming here
\end{itemize}

\var{fromPage} is a string containing the URL of the page the user was trying to access.

\var{login} is a string containing the login of the user if any. If the string is not empty, it means that
 the user already entered a login, but the password was incorrect, or that the user had a cookie with the login in it.
This allows to display the login in the form so the user doesn't have to enter it each time.

The CherryClass comes with a default \var{loginScreen} mask. You'll probably want to overwrite it to customize it for
your needs. All you have to do is define a form that calls the \var{doLogin} method with 3 parameters:
\var{login}, \var{password} and \var{fromPage}. The first two are entered by the user. The third one should be a hidden
field with the value that's passed to the function.

The following code is the default implementation of the \var{loginScreen} mask:
\begin{verbatim}
<html><body>
    Message: <div py-eval="message">message</div>
    <form method="post" action="doLogin">
        Login: <input type=text name=login py-attr="login" value="" length=10><br>
        Password: <input type=password name=password length=10><br>
        <input type=hidden name=fromPage py-attr="fromPage" value=""><br>
        <input type=submit>
    </form>
</body></html>
\end{verbatim}
\end{funcdesc}

\begin{funcdesc}{mask or view: logoutScreen}{}
This page is displayed after the user logged out. This method is called by the \var{doLogout} method.
You may overwrite it to suit your needs.
\end{funcdesc}

\begin{funcdesc}{view: doLogout}{}
This is the mask or view you should call to perform a logout. This method performs the logout, and then calls
the \var{logoutScreen} method to display the logout screen.
\end{funcdesc}

\begin{memberdesc}{variable: login}
String containing the login of the user that is logged in. The string is empty if no-one is logged in.
\end{memberdesc}

\begin{seealso}
  \seemodule{CookieSessionAuthenticate}{Cookie/session-based authentication.}
  \seemodule{HttpAuthenticate}{Basic HTTP authentication.}
\end{seealso}


\subsection{Example}
The following code is an exemple that uses the CookieAuthenticate module:

\begin{verbatim}
use CookieAuthenticate

CherryClass MemberArea(CookieAuthenticate):
mask:
    def index(self):
        <html><body>
        Welcome to the member area, <py-eval="self.login"><br>
        If you want to log out, just click <a py-attr="self.getPath()+'/doLogout'" href="">here</a>.<br>
        Otherwise, just click <a py-attr="request.base" href="">here</a> to go back to the home page.
        </body></html>
    def loginScreen(self, message, fromPage, login=''):
        <html><body>
            Welcome to the login page. Please enter your login and password below:
            <py-if="message==self.wrongLoginPasswordMessage">
                <br><font color=red>Sorry, the login or password was incorrect</font>
            </py-if>
            <form method="post" action="doLogin">
                Login: <input type=text name=login py-attr="login" value="" length=10><br>
                Password: <input type=password name=password length=10><br>
                <input type=hidden name=fromPage py-attr="fromPage" value=""><br>
                <input type=submit value="Login">
            </form>
        </body></html>
    def logoutScreen(self):
        <html><body>
            You have been logged out.<br>
            Click <a py-attr="request.base" href="">here</a> to go back to the home page.
        </body></html>
function:
    def getPasswordListForLogin(self, login):
        if login=="login": return ["password"]
        return []

CherryClass Root:
mask:
    def index(self):
        <html><body>
            Welcome to the site.<br>
            Click <a href='memberArea/index'>here</a> to access the
            member area.
        </body></html>
\end{verbatim}


