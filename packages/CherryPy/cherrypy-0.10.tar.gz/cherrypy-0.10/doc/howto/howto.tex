\documentclass{manual}
\usepackage[T1]{fontenc}

\title{CherryPy HowTo}

\author{Remi Delon}
\authoraddress{Email: \email{remi@cherrypy.org}}

\date{\today}			% XXX update before release!
\release{0.10}			% software release, not documentation
%\setreleaseinfo{a0}		% empty for final release
%\setshortversion{2.3}	% major.minor only for software


\begin{document}

\maketitle

\tableofcontents

\chapter{How to serve gzip-compressed pages with CherryPy}

Most browser support gzip-compression. If you have large HTML pages (40K or more),
compressing them can really make a big difference (a 50K page might only be 5-6K
when it's compressed).

Doing that with CherryPy is really easy. It only takes a few lines of code and uses
the \var{initNonStaticResponse} special function:
\begin{verbatim}
import gzip, cStringIO

def initNonStaticResponse():
    # Only compress the page if the client said it accepted it
    if request.headerMap.get('accept-encoding', '').find('gzip')!=-1:
        # Compress page
        zbuf=cStringIO.StringIO()
        zfile=gzip.GzipFile(mode='wb', fileobj=zbuf, compresslevel=9)
        zfile.write(response.body)
        zfile.close()
        response.body=zbuf.getvalue()
        response.headerMap['content-encoding']='gzip'
\end{verbatim}


\chapter{How to run a CherryPy server behind Apache}

\section{Introduction}

CherryPy's built-in HTTP server is now pretty robust, but some people might still want to run CherryPy behind Apache.

Whether to run CherryPy exposed or behind Apache depends on many criteria so this question is out of the scope of this HowTo.

This HowTo will show you how to run CherryPy behind Apache, but you can probably adapt it for any other
webserver.

There are several ways to run CherryPy behind Apache:

\section{Using persistent CGI}

The way it works is very easy to understand. We use a small cgi script called \strong{cherrypcgi.cgi}
as a link between Apache and CherryPy. When someone requests a page, Apache invokes this cgi script.
The script then connects to the CherryPy server, gets the page and returns it to Apache.

Of course, it takes a little extra time because a new process is created everytime the script is
called. And also because the request and the response go through the script instead of going directly
from Apache to CherryPy and back.

But that extra time is really small, so it's not a big problem. (and if you have a really high traffic
website, you can use load balancing anyway :-)

\subsection{Example}
The cgi script uses the HTTP protocol to talk to the CherryPy server. Therefore, the CherryPy server
doesn't make any difference between talking to a browser or talking to the cgi script.

This means that you can just start it normally.

If you're on Unix, it's probably better to run the CherryPy server on an AF_UNIX socket. This is what we will
do in this example.

To do so, edit the configuration file of the CherryPyServer and enter the following lines:
\begin{verbatim}
[server]
socketFile=socketFile.soc
\end{verbatim}
This means that the CherryPy server will listen on that AF_UNIX socket instead of a regular socket port.

Now, just edit the \file{cherrypcgi.cgi} file provided with the distribution and modify the line that says:
\begin{verbatim}
socketFile='put your socket file here'
\end{verbatim}
Put the name of the AF_UNIX socket where the CherryPy server is listening.

The last thing to do is to configure Apache so it will call the cgi script for each request:
In Apache's configuration file (\file{commonhttpd.conf} for instance), add the following lines (of course, you have
to adapt the path of cherrypcgi.cgi):
\begin{verbatim}
RewriteEngine on
RewriteRule ^(.*) /home/cherrypy/cherrypcgi.cgi$1 [e=HTTP_CGI_AUTHORIZATION:%1,t=application/x-httpd-cgi,l]
\end{verbatim}
And that's it ! Start Apache, start the CherryPy server and it should work.

It is also possible to tweak cherrypcgi.cgi so it automatically restarts the CherryPy server if it ever goes down.
There will be another HowTo on this.

\section{Using FastCGI}

FastCgi works very much like persistent CGI, except the cgi script is constantly running. This means that no process
has to be created for each request, which saves a lot of time !

The current implementation of FastCGI in CherryPy is not optimized because the FastCGI script is a standalone script
that connects to the CherryPy backend, instead of being directly integrated in the backend. But this method is still a
lot faster than persistent CGI.

The FastCGI script is called \strong{cherryfcgi.cgi}

\subsection{Example}
The FastCGI script uses the HTTP protocol to talk to the CherryPy server. Therefore, the CherryPy server
doesn't make any difference between talking to a browser or talking to the FastCGI script.

This means that you can just start it normally.

If you're on Unix, it's probably better to run the CherryPy server on an AF_UNIX socket. This is what we will
do in this example.

To do so, edit the configuration file of the CherryPyServer and enter the following lines:
\begin{verbatim}
[server]
socketFile=socketFile.soc
\end{verbatim}
This means that the CherryPy server will listen on that AF_UNIX socket instead of a regular socket port.

Now, just edit the \file{cherryfcgi.cgi} file provided with the distribution and modify the line that says:
\begin{verbatim}
socketFile='put your socket file here'
\end{verbatim}
Put the name of the AF_UNIX socket where the CherryPy server is listening.

The last thing to do is to configure Apache so it will connect to the FastCGI script for each request:
In Apache's configuration file (\file{commonhttpd.conf} for instance), add the following lines (of course, you have
to adapt the path of cherrypcgi.cgi):
\begin{verbatim}
SetHandler fastcgi-script
RewriteEngine on
RewriteRule ^(.*) /home/cherrypy/cherryfcgi.cgi/$1 [L]
\end{verbatim}
You also need to make sure that Apache loads the FastCGI module. The following lines should be somewhere in your Apache configuration files:
\begin{verbatim}
LoadModule fastcgi_module    modules/mod_fastcgi.so
AddModule mod_fastcgi.c
\end{verbatim}

And that's it ! Start Apache, start the CherryPy server and it should work.

\begin{seealso}
    \seeurl{http://www.fastcgi.com}{For more information on FastCGI}
\end{seealso}

\section{Using mod_rewrite}

It's easy to configure Apache so it just passes all requests to the CherryPy server, reads the response and
passes the response to the client.

This method is a bit faster than persistent CGI because no CGI process needs to be created for each request.

\subsection{Example}
In this example, we'll assume that the web site is www.cherrypy.org and that the CherryPy server is running
on the same machine as Apache, on the port 8000. Of course, you'll have to adapt it for your own configuration.

Configuring Apache is very easy:
\begin{verbatim}
RewriteEngine on
RewriteRule ^(.*) http://localhost:8000$1 [p]
\end{verbatim}

The last thing we have to do is tell CherryPy that it's serving pages for www.cherrypy.org (and not localhost).
All it takes is 3 lines of code in the \var{initRequest} special function:

\begin{verbatim}
def initRequest():
    request.headerMap['host']='www.cherrypy.org'
    request.base='http://www.cherrypy.org'
    request.browserUrl=request.browserUrl.replace('http://localhost:8000', 'http://www.cherrypy.org')
\end{verbatim}

And voila !

Note, this can also be done with mod_proxy instead of mod_rewrite.

Many thanks to Andreas Kostyrka for this tip.


\chapter{How to connect a CherryPy server to a database}
Many databases already have a corresponding Python module: Oracle, Sybase, MySql, PostgreSql, ...

Accessing these databases from a CherryPy server is really easy. All you have to do is use the \var{initServer}
function to connect to the database.

For instance, you can connect to a MySql database using the following code (it uses the MySQLdb Python module):
\begin{verbatim}
import MySQLdb
def initServer():
    global dbConnection
    dbConnection=MySQLdb.connect(host, user, passwd, schema)
\end{verbatim}
The following code is an example of running a query and using the result to build a page:
\begin{verbatim}
CherryClass Root:
view:
    def index(self):
        # Run the MySql query
        cursor=dbConnection.cursor()
        cursor.execute("select count(*) from user")
        result=cursor.fetchall()
        cursor.close()

        page="<html><body>"
        nbUser=result[0][0]
        page+="There are %s users in the database"%nbUser
        page+="</body></html>
        return page
\end{verbatim}

This example can easily be adapted for any kind of database. Just find the appropriate Python module for
the database and read its documentation.

\chapter{How to use load-balancing for your web site}
\section{Introduction}
Having a lot of traffic on their web site is what most webmasters dream about ...

Unfortunately, it also comes with its share of problems.
Basically, if requests come in faster than your server can handle them, you're screwed :-)

One way to deal with that is to use load-balancing. This basically means that several threads or processes will
be serving the pages. It is only efficient if you have several machines or if your machine has several processors.
(otherwise, the same processor will just run several threads or processes, but the overall speed will be the same).

Of course, this means that you have to take care of the data sharing between the threads/processes. One way to
do that is to use a database where you store all the data that needs to be shared amongst the processes. All processes
read and write to the same database, insuring that they all have the same informations.

There are two ways to do load-balancing with CherryPy. One of them is easier to set up, but only applies to
multi-processor machines running a Unix-based OS.

\section{Generic load-balancing method}
Since a CherryPy server is a self-contained process that includes everything to run the web site, it is easy to
start as many processes as you want (on the same machine or on different machines). If you start several
processes on the same machine, you just have to use a different configuration file for each of them to specify
a different port each time.

Then all you have to do is use a simple load-balancer (there are many of those out there) to redirect the requests
to your CherryPy servers.


Let's take an example:
\begin{itemize}
\item
You have 3 machines, called host1, host2 and host3
\item
host2 and host3 are single-proc, and host1 is dual-proc.
\item
For the load-balancer, we'll use \strong{balance}, an easy to use, lightweight, open source load balancer available from
\url{http://balance.sourceforge.net}.
\item
On host1, we'll run 2 CherryPy servers (since it's dual-proc) plus the load-balancer
\item
On host2 and host3, we'll only run one CherryPy server.
\item
\strong{balance} will listen on port 80. The CherryPy servers on host1 will listen on ports 8080 and 8081. The CherryPy servers on
host2 and host3 will listen on port 8080.
\item
Let's assume that your CherryPy server file is called \file{MyServer.py}
\end{itemize}

Here is what we have to do:
\begin{itemize}
\item
Copy \file{MyServer.py} on host1, host2 and host3. (alternatively, you can export the file system from one machine to the others, to deal with only one copy of the file)
\item
On host1, host2 and host3, create a configuration file called \file{MyServer8080.cfg} containing the following:
\begin{verbatim}
[server]
socketPort=8080
\end{verbatim}
On host1, create another configuration file called \file{MyServer8081.cfg} containing the following:
\begin{verbatim}
[server]
socketPort=8081
\end{verbatim}
\item
On host1, host2 and host3, start the servers that listen on port 8080:
\begin{verbatim}
[host1] %  python MyServer.py -C MyServer8080.cfg
[host2] %  python MyServer.py -C MyServer8080.cfg
[host3] %  python MyServer.py -C MyServer8080.cfg
\end{verbatim}
\item
On host1, start the second server that listens on port 8081:
\begin{verbatim}
[host1] %  python MyServer.py -C MyServer8081.cfg
\end{verbatim}
\item
Now all you have to do is start the load balancer and indicate what the available CherryPy servers are:
\begin{verbatim}
[host1] % /usr/sbin/balance 80 host1:8080 host1:8081 host2:8080 host3:8080
\end{verbatim}
\end{itemize}
And voila !

\section{Multi-processor, unix-based machine}
The method described in the previous section works in all cases, but if you have a unix-based machine with multiple
processors, there is another method for load-balancing with CherryPy.

The trick is to create the socket where the CherryPy server will listen, and then do a \strong{fork()}. This way, we'll
have multiple processes listening on the same socket. When one process is busy building a page, the next one will be
listening on the socket and thus serving a request that might come in.

This feature is built in. All you have to do to use it is to use the \strong{fixedNumberOfProcesses} option in
the configuration file (in the \var{[server]} section):
\begin{verbatim}
[server]
socketPort=80
fixedNumberOfProcesses=3
\end{verbatim}
And that's it.

Note: This method is only useful if your machine has several processors. If not, then the processor will run multiple
CherryPy processes, but the overall speed won't be improved.

\chapter{How to compile your code in debug mode}
One way to debug your code is to add \code{print} statements in your code. The output will show up in the window
where the CherryPy server runs. However, this method may not always work (for instance, if the server is running
as a daemon on a production machine and you don't have access to its standard output). Plus switching from debug mode
to non-debug mode is not easy (you have to comment out all the \code{print} statements ...).

Another way is to save this output in a log file. You can redirect the standard output from the server, or you can write
in the log file directly from your code. CherryPy provides one special function, one CGTL tag and one CHTL tag to do that.

The special function is called \var{debug} and the tag is called \var{py-debug}. They are used like this:
\begin{verbatim}
debug('i=%s'%i)
<py-debug="'i=%s'%i">
<div py-debug="'i=%s'%i"></div>
\end{verbatim}
All these statements have the same effect: add a line in the log file.

If the name of the CherryPy server is \file{RootServer.py}, the name of the log file is \file{RootServer.log}.

Debugging can be easily turned on or off at compile time, using the \strong{-D} option of the compiler.
If \strong{-D} is specified, then debugging is turned on:
\begin{verbatim}
python ../cherrypy.py -D Root.cpy
\end{verbatim}
If \strong{-D} is omitted, debugging is turned off.

CherryPy uses a special global variable called \var{_debug} that can be accessed from anywhere in your code. The variable
is 1 if debugging is on, 0 otherwise.


\chapter{How to use the hotReload feature of CherryPy}
\section{Introduction}
The development cycle of a CherryPy web site is the same as the development cycle of a regular software:
\begin{itemize}
\item
Write source files
\item
Compile them
\item
If compiler generated some errors, fix source files and try to compile again
\item
Once compilation was OK, stop running executable and restart the newly generated executable
\item
If we find some bugs or we want some new features, modify the source files and go back to the compilation stage
\end{itemize}

The compilation process is usually pretty fast (if you have a "decent" computer).

But what if the CherryPy server has a long initialization process (maybe it needs to connect to several ressources,
like databases, or it needs to precompute some data). All this work is usually done in the \var{initServer} special
function. If this initialization takes a lot of time, having to stop and restart the server everytime you make
a change is really a loss of time while developing your web site.

This is where the \var{hotReload} feature comes in. It allows you to refresh your code without having to stop
the server.


\section{How does it work ?}
When it receives a \var{hotReload} request, the server will just read its source code again, updating all the CherryClasses
code and recreating new CherryClasses instances with the new code. It will not execute \var{initServer} again. Instead,
it will execute \var{hotReloadInitServer} if any.

\section{How to use it ?}

The \var{hotReload} feature is enabled in the executable when the file is compiled with the \var{-D} flag (debug mode).

To send a \var{hotReload} request to a running CherryPy server, just use the little script called \file{cherryhotreload.py}
provided with the distribution.

Let's take a practical example that shows how to use the \var{hotReload} feature:
\begin{itemize}
\item
You write a first version of your source file called \file{Root.cpy}
\item
You compile your source file in debug mode:
\begin{verbatim}
python ../cherrypy.py -D Root.cpy
\end{verbatim}
\item
In a window, you start the CherryPy server:
\begin{verbatim}
python RootServer.py
\end{verbatim}
\item
Now you have a running CherryPy server
\item
You realize that you want to change something in your code
\item
Just update your source file and recompile it (in another window, since the server is still running in the first window)
\item
Send the \var{hotReload} request to the running server to tell it to refresh its code:
\begin{verbatim}
python cherryhotreload.py localhost:8080
\end{verbatim}
(This assumes that the server is running on localhost:8080. Of course, you have to adapt it for your own configuration)
\item
In the window where the server is running, you see a message saying that it has performed the \var{hotReload}
\end{itemize}
Warning: After a \var{hotReload}, the line numbers that the python interpreter might give you in case of an error
may not match your source file. This happens if you add or remove lines of code before the \var{hotReload}

\chapter{How to use caching}
\section{Introduction to caching}
If you have a web site where pages take a long time to build but don't change too often, then caching is just the tool you
need to speed up your web site.

Here is how caching works:
\begin{itemize}
\item
The first time someone requests a page, the server builds the page normally and returns the page to the client
\item
The server also saves a copy of the page "somewhere" for later retrieval. It also sets a delay for how long this copy
can be used.
\item
The next time someone requests the same page, the server uses the copy it made instead of rebuilding the page.
\item
When the copy expires, the server builds a new version of the page.
\end{itemize}

\section{Caching with CherryPy}
\subsection{Where are pages stored ?}
With CherryPy, all pages are saved in memory. This allows for maximum speed, but it also means that the size of
your process will grow really fast if you cache lots of big pages. To avoid that, CherryPy provides a way
to "purge" old pages that are in the cache.

\subsection{How does it know if a page is already in the cache ?}
The page that a server returns to a client usually depends on 3 parameters:
\begin{itemize}
\item
The URL that the client requested
\item
The parameters that the client sent (via a GET or a POST)
\item
The cookies that the client sent
\end{itemize}
CherryPy lets you define what your cache "key" will be. Two requests that have the same cache key will receive the
same response from the server.

For instance, if your pages don't depend on cookies, your cache key could just be \var{request.browserUrl}.
But if your pages depend on cookies, your cache key can be \var{request.browserUrl+str(request.simpleCookie)}.

\subsection{How do I control which pages I want to cache or not ?}
It is very easy. All you have to do is use \var{initRequest} or \var{initNonStaticRequest} to set a couple of
special variables if you want to use caching:
\begin{itemize}
\item
\var{request.cacheKey}:
this is the cache key as described in the previous section
\item
\var{request.cacheExpire}:
This is the time the cached version of the page will expire
\end{itemize}
Let's take an example:
\begin{itemize}
\item
You have a web site with 3 main parts: /part1, /part2 and /part3
\item
Pages under /part1 are complicated to build, and the pages might change every 30 minutes: we'll use caching on this part and
the caching delay will be set to 10 minutes.
\item
Pages under /part2 are complicated to build, and the pages might change every day: we'll use caching on this part and the
caching delay will be set to 12 hours.
\item
Pages under /part3 are very fast to build, and not many people visit this area: we won't use caching on this part
\item
Pages only depend on the URL, not on cookies: we'll use the URL as the cache key
\end{itemize}
Here is what the \var{initNonStaticRequest} special function will look like:
\begin{verbatim}
import time

def initNonStaticRequest():
    if request.path.find('part1')==0:
        request.cacheKey=request.browserUrl
        request.cacheExpire=time.time()+30*60 # 30 minutes
    elif request.path.find('part2')==0:
        request.cacheKey=request.browserUrl
        request.cacheExpire=time.time()+12*60*60 # 12 hours
\end{verbatim}
That's all ! Just restart your server and the pages will be cached

\subsection{How do I control when the cache is purged ?}
Since all cached pages are stored in memory, the memory usage of the server will grow really fast if it has
to cache lots of different pages (or, to be more precise, "different cache keys"), especially if these pages
are big. For this reason, you should only cache
pages for which it will really make a speed difference (good candidates are pages that are complicated to
build but that don't change too often).

If all your pages in the cache are requested very often, there isn't much you can do to lower the memory usage.

But if some of them are only requested once in a while, then it might be worth it to remove them from the cache
when they haven't been requested for a long time. This will free up some memory.

To control how often CherryPy will purge the cache, use a parameter called \var{flushCacheDelay} in the section \var{cache}
of the config file:
\begin{verbatim}
[cache]
flushCacheDelay=10 # In minutes
\end{verbatim}
The default value for the \var{flushCachDelay} is zero, which means that the cache will never be flushed.

Note: This parameter is only useful in very specific cases, so you probably don't need to use it ...

\chapter{How can webdesigners and webdevelopers collaborate on a CherryPy project}
\section{Introduction}
The two common kinds of people involved in a website development are designers and developers. Designers typically take care
of the presentation of the pages, and developers take care of the content and the logic behind the web site.

Designers and developers don't use the same tools: designers usually use an HTML editor (for instance, Dreamweaver or Amaya)
and developers use a text editor and CherryPy.

\section{How can they collaborate ?}
It's very easy:
\begin{itemize}
\item
Designers create template files with their HTML editor
\item
Developers edit the template files and add some dynamic data if needed (using CHTL)
\item
Since CGTL is HTML-editor-safe, designers can re-edit the template files and add new HTML code, without losing the dynamic
information.
\item
Developers include the template files in their CherryPy source code using the <py-include> tag in their masks:
\begin{verbatim}
<div py-include="header.html">header</div>   (CHTL form)
or
<py-include="header.html"> (CGTL form)
\end{verbatim}
\end{itemize}

The template file will be included in the code by the CherryPy preprocessor (very much like an "\#include" directive in C).
To find the template files, the preprocessor will look in the same directories as for the other source files
(\file{/lib}, \file{/src}, directories specified with the -I flag, etc.)

\section{Example}
In this example, we'll build a small website with only 2 pages. Each page will be made of a header, a body and a footer.
The header and the footer will be the same on both pages, except they will have a "you are here" label.

Let's go step by step
\begin{itemize}
\item
The webdesigners create 4 files with their favorite HTML editor:

\begin{verbatim}
*******************
* File header.html
*******************
<html><body>
Welcome to the CherryPy py-include demo.<br>
You are here: home<br><br>

*******************
* File footer.html
*******************
<br><br><small>Don't forget to eat cherry pie every day</small>
</body></html>

*******************
* File page1.html
*******************
Header

Welcome to page1.<br>
Click <a href="page2">here</a> to go to page 2.

Footer

*******************
* File page2.html
*******************
Header

Welcome to page2.<br>
Click <a href="page1">here</a> to go to page 1

Footer
\end{verbatim}
Nothing too fancy so far. We just have a header, a footer, and two bodies. Everything is static.

\item
At this point, the webdevelopers will edit the html files and add some dynamic information:
\begin{itemize}
\item
In header.html, change 'home' into '<div py-eval="youAreHereLabel">home</div>'
\item
In page1.html and page2.html, change 'Header' into '<div py-include="header.html">Header</div>' and 'Footer' into
'<div py-include="footer.html">Footer</div>'
\end{itemize}

\item
Now the templates are ready, and the webdevelopers can write their CherryPy source code:
\begin{verbatim}
CherryClass Root:
mask:
    def page1(self, youAreHereLabel='page1'):
        <py-include="page1.html">
    def page2(self, youAreHereLabel='page2'):
        <py-include="page2.html">
\end{verbatim}

We now have a working version of the web site: just compile the CherryPy source file, run the server and test it.

\item
Now, let's assume that the designers want to change something in the look of the page: they want the "CherryPy" word in the
header to be displayed in bold . To do so, all they have to do is edit the "header.html" file and put the "CherryPy" word
in bold. The HTML editor should leave the '<div py-eval="youAreHere">home</div> untouched. (it keeps the dynamic information).

\end{itemize}

\chapter{How to use SSL with CherryPy}

\section{Introduction}
There are several ways to make an SSL CherryPy website:
\begin{itemize}
\item
Either by running the CherryPy server behind Apache and by configuring Apache to handle SSL
\item
Or by running the CherryPy server directly and by configuring CherryPy to handle SSL
\end{itemize}
Only the second option is covered in this HowTo.

\section{Prerequisite}
CherryPy's SSL support is based on the PyOpenSSL package. You can get it at \url{http://pyopenssl.sourceforge.net}.
I recommend using version 0.5.1 or higher of PyOpenSSL. Problems have been reported with earlier versions on some platforms.

You can test your installation by firing up the Python interpreter and typing:
\begin{verbatim}
>>> import OpenSSL
>>>
\end{verbatim}
If this doesn't work for you, then PyOpenSSL is not correctly installed on your system.

\section{Configuring the CherryPy server}
Once you have PyOpenSSL installed, all you have to do is add 2 lines in your CherryPy config file, in the \var{server} section:
\begin{verbatim}
sslKeyFile=/path/to/ssl/key/file
sslCertificateFile=/path/to/ssl/certificate/file
\end{verbatim}

And that's it !

A "real-world" sample config file for an SSL site might be:
\begin{verbatim}
[server]
socketPort=443
sslKeyFile=server.pkey
sslCertificateFile=server.cert
\end{verbatim}

\chapter{How to use XML/XSL with CherryPy}

\section{Introduction}
With CherryPy, you can use python standard datatypes (lists, dictionaries, ...) to store your data, and then you can use
masks (written in CHTL) to convert that data into HTML.

However, this can also be done with XML and XSL. I personally think that using python types directly, and then writing
CHTL masks is much easier and straightforward than XML/XSL, but there might be several reasons why you'd want to use XML/XSL:
\begin{itemize}
\item
You're getting your data from some other application and that data comes in XML (which is likely to
happen, since XML/XSL is now a standard).
\item
Your website is a multi-million contract with a big company, and they want to hear words like "XML" or "XSL", because
it makes them feel safe ...
\item
Your boss will let you use CherryPy, but only if you use XML/XSL, because it makes him feel safe ...
\end{itemize}

\section{Prerequisite}
This HowTo uses the 4Suite module developed by Fourthought, Inc. I tested it with version 0.12.0a1 of 4Suite and version
2.1 of python, but other combinations would also probably work.

Also, several other XML/XSL modules are available for Python, and using them is also probably very easy.

Start by downloading 4Suite from \url{http://4suite.org} and installing it on your machine.

You can test your installation by firing up the Python interpreter and typing:
\begin{verbatim}
>>> from Ft.Xml.Xslt.Processor import Processor
::: Using pDomlette
>>>
\end{verbatim}
If this doesn't work for you, then 4Suite is not correctly installed on your system.

\section{Using the XML/XSL package from CherryPy}

Once you have 4Suite installed, is is very easy to use it from CherryPy. You can use regular masks to write your XSL
stylesheets, and you have several ways to generate your XML (using a function, a view or even a mask).

The following example is a simple example that demonstrates a basic XML/XSL transformation:

\begin{verbatim}
from Ft.Xml.Xslt.Processor import Processor

CherryClass XslTransform:
function:
    def transform(self, xslStylesheet, xmlInput):
        processor = Processor()
        processor.appendStylesheetString(xslStylesheet)
        return processor.runString(xmlInput, 0, {})

CherryClass Root:
view:
    def index(self):
        return xslTransform.transform(self.xslStylesheet(), self.xmlInput())
mask:
    def xslStylesheet(self):
        <?xml version="1.0" encoding="ISO-8859-1"?>

        <xsl:stylesheet version="1.0"
            xmlns:xsl="http://www.w3.org/1999/XSL/Transform">

        <xsl:template match="/">
            <html><body>
                <h2>My CD Collection</h2>
                <table border="1">
                    <tr bgcolor="#9acd32">
                        <th align="left">Title</th>
                        <th align="left">Artist</th>
                    </tr>
                    <xsl:for-each select="catalog/cd">
                        <tr>
                            <td><xsl:value-of select="title"/></td>
                            <td><xsl:value-of select="artist"/></td>
                        </tr>
                    </xsl:for-each>
                </table>
            </body></html>
        </xsl:template>

        </xsl:stylesheet> 
    def xmlInput(self):
        <?xml version="1.0" encoding="ISO-8859-1"?>
        <catalog>
            <cd>
                <title>Empire Burlesque</title>
                <artist>Bob Dylan</artist>
            </cd>
            <cd>
                <title>Hide your heart</title>
                <artist>Bonnie Tyler</artist>
            </cd>
        </catalog>
\end{verbatim}

How does it work ?
\begin{itemize}
\item
\var{xslStylesheet} is a mask containing the XSL stylesheet
\item
\var{xmlInput} is a mask containing the XML input. In real life, this XML would probably be generated from a function,
based on data coming from a database or some other ressource.
\item
The \var{index} view just says to apply this XSL stylesheet to that XML input.
\item
The \var{XslTransform} CherryClass is just used to implement the \var{transform} function, to avoid having to repeat
these 3 lines of code for each transformation.
\end{itemize}

Here is the HTML document generated in this example:
\begin{verbatim}
<html>
  <body>
    <h2>My CD Collection</h2>
    <table border='1'>
      <tr bgcolor='#9acd32'>
        <th align='left'>Title</th>
        <th align='left'>Artist</th>
      </tr>
      <tr>
        <td>Empire Burlesque</td>
        <td>Bob Dylan</td>
      </tr>
      <tr>
        <td>Hide your heart</td>
        <td>Bonnie Tyler</td>
      </tr>
    </table>
  </body>
</html>
\end{verbatim}

PS: Note that it seems that the 4Suite API has changed again in the latest releases, so you might have to tweak the above code if you're using a recent release.

\chapter{How to use AOP (Aspect Oriented Programing) with CherryPy}

\section{Introduction}
This HowTo is intended to people who are already familiar with OOP, but who are not necessarily familiar with AOP.
If this is your case, check out \url{http://aosd.net} for an introduction to AOP.

CherryPy only implements a few of AOP concepts. I don't have a PHD in AOP and it is not my goal to make
CherryPy an AOP reference :-)

Instead, I implemented a few features that I thought might be useful (and they already are, since both
HttpAuthenticate and CookieAuthenticate modules use AOP features).

\section{Basic example}
On most websites a common header and footer is used for all (or most) pages of the site. In CherryPy, you
can easily implement that using a "classic" object oriented approach: first you define a class that contains
the header and the footer that you want to use for all your pages.

\begin{verbatim}
CherryClass NormalHeaderAndFooter:
mask:
    def header(self):
        <html><body>I'm the header<br><br>
    def footer(self):
        <br><br>I'm the footer</body></html>
\end{verbatim}

Then, all you have to do is derive your CherryClasses from \var{NormalHeaderAndFooter} and call \var{header}
and \var{footer} in each of your masks or views:

\begin{verbatim}
CherryClass Root(NormalHeaderAndFooter):
mask:
    def index(self):
        <py-eval="self.header()">
            Hello, world !
        <py-eval="self.footer()">
    def otherPage(self):
        <py-eval="self.header()">
            I love cherry pie !
        <py-eval="self.footer()">
\end{verbatim}

That's the "classic" object oriented approach. It's pretty good, but if we look at it, we'll realize that we
have to repeat \var{<py-eval="self.header()">} and \var{<py-eval="self.footer()">} for each mask and view, and this
has several drawbacks:

\begin{itemize}
\item
First of all, I \strong{never} like to repeat code. Whenever I do it, I feel like there must be something
I'm doing wrong and my code could certainly be improved !
\item
For a real life website, we probably would have many CherryClasses that inherit from \var{NormalHeaderAndFooter}, and all these
CherryClasses would have many masks that all make the same call to \var{header} and \var{footer}. What if we want to change
the name of the \var{header} method ? In that case, we would have to go through all the masks of all the CherryClasses to
change the line !
\item
In this trivial example, we only have 2 lines of code to repeat, but what if we had other things to do for each page
(like log the request in a database or check that the user is authenticated) ? We could end up repeating tens of lines
of code for each method !
\end{itemize}

That's where AOP comes in ... Instead of just declaring "normal" \var{header} and \var{footer} methods, we'll declare
"aspect" methods. This basically adds another information which is: "include my code at the beginning or at the end
of each of the methods of all derived classes".

Therefore, the implementation of our simple example using AOP would be:

\begin{verbatim}
CherryClass NormalHeaderAndFooter:
aspect:
    (1) start:
        _page.append("<html><body>I'm the header<br><br>")
    (1) end:
        _page.append("<br><br>I'm the footer</body></html>")

CherryClass Root(NormalHeaderAndFooter):
mask:
    def index(self):
            Hello, world !
    def otherPage(self):
            I love cherry pie !
\end{verbatim}
\footnote{This sample code requires CherryPy-0.7 or later to work}

See ... we got rid of the \var{<py-eval="self.header()">} and \var{<py-eval="self.footer()">} lines ...

Let's see what we can notice from looking at the code:
\begin{itemize}
\item
In CherryClass \var{NormalHeaderAndFooter}, the "mask" section has disappeared and has been replaced by a new "aspect" section.
\item
In this "aspect" section, the line where we normally declare methods (using the "def" keyword) has been replaced by a special
line that contains "start" or "end" (we'll explain the syntax later).
\item
The lines that normally contain the body of the function are still there, and they're written using regular Python.
\item
\var{<py-eval="self.header()">} and \var{<py-eval="self.footer()">} have disappeared from the methods of \var{Root}.
\var{Root} is still derived from \var{NormalHeaderAndFooter}.
\end{itemize}

Here is how it works:

The header for an aspect method has two parts:
\begin{verbatim}
(condition) startOrEnd:
\end{verbatim}
\var{startOrEnd} can be either the "start" or the "end" keyword. This indicates whether the code should be appended at
the start or at the end of the methods.

\var{condition} is a python expression used to indicated which method the aspect should apply to. If the condition
is always true (as in "1"), then it means that the aspect will apply to all methods of the derived CherryClasses. The
python expression can use a special variable called \var{method} that contains the following member variables:
\begin{itemize}
\item
\var{name}: contains the name of the method
\item
\var{type}: contains the type of the method ("mask", "view", "variable", ...)
\item
\var{isHidden}: is true if the mask or view is hidden (false otherwise)
\item
\var{className}: name of the CherryClass the method belongs to
\end{itemize}

Let's go back to our example: let's imagine that we want to use the regular header and footer for both
pages \var{index} and \var{otherPage}, but we want a third page called \var{yetAnotherPage} that has its own header
and footer.
Here is what the code would look like:
\begin{verbatim}
CherryClass NormalHeaderAndFooter:
aspect:
    (method.type=='mask' and method.name!='yetAnotherPage') start:
        _page.append("<html><body>I'm the header<br><br>")
    (method.type=='mask' and method.name!='yetAnotherPage') end:
        _page.append("<br><br>I'm the footer</body></html>")

CherryClass Root(NormalHeaderAndFooter):
mask:
    def index(self):
            Hello, world !
    def otherPage(self):
            I love cherry pie !
    def yetAnotherPage(self):
        <html><body bgcolor=red>
            I love cherry pie !
        </body></html>
\end{verbatim}
As we can see, we use the aspect condition to indicate which method the aspect will apply to. Note that
the "method.type='mask'" condition is useless, since all methods of \var{Root} are masks. So it is
only there to show you an example of an aspect condition.

\section{How is it used in CookieAuthenticate and HttpAuthenticate}
Now that you know how it works, you can look at the source code for these two modules and you should be able to
understand it.

Basically, when you declare a CherryClass that inherits from CookieAuthenticate or HttpAuthenticate, some code is
automatically added at the beginning of each of your masks and views. This code checks if the user is authenticated
or not. If not, it returns the login page instead of the regular page ...

\chapter{How to create a spinning server and then debug it}

\section{Creating a spinning server}
By "spinning" server, I mean a server that's stuck in a loop while trying to build a page.

The easiest way to achieve that is:
\begin{verbatim}
CherryClass Root:
view:
    def index(self):
        while 1: pass
\end{verbatim}
Compile that, start the server and try to request the home page ... :-) There you have it: a nice "spinning" server. Of course,
the page will never be rendered, and if you try to load the same page from another browser, your browser won't even be able to
connect to the CherryPy server because it's still busy trying to serve the first page (unless you have multiple server
processes running).

The way you can tell that your server is spinning is very easy: you're no longer able to connect to it, and the cpu time used
by the process (which is displayed when you do a \var{ps} on Unix for instance) keeps growing, meaning that the
process is not stalled, but it's actually doing something.

You might be thinking "I'm not that stupid ! I will never write such an obvious loop !".

But here is another less trivial case that actually happened to me (and that's the reason why I wrote this HowTo):
\begin{verbatim}
CherryClass Root:
function:
    def extractText(self, code):
        # Remove text between < and >
        while 1:
            i=code.find('<')
            if i==-1: break
            j=code.find('>', i)
            if j!=-1: code=code[:i]+' '+code[j+1:]
        return code
mask:
    def index(self, code=''):
        <html><body>
            Extracted text: <py-eval="self.extractText(code)">
            <form action="index">
            New text: <textarea name="code"></textarea>
            <input type=submit>
            </form>
        </body></html>
\end{verbatim}
The \var{extractText} function takes some HTML code as an input and strips out the HTML tags. The output is the text
that's extracted from the HTML code. The \var{index} mask just provides a textarea-based interface to test the function.

If you compile this code, run it and test it with some HTML code, it might run fine for a while, but in some cases it will
start "spinning" ! The reason is that the \var{extractText} function is buggy in some cases: if the code is not proper
HTML (for instance, a tag is opened with a "<" but never closed with a ">"), the function will enter an infinite loop.
Correcting that is very easy once you know what's happening (just add "else: break" after the second "if" of
the function).

But when this is happening on your production server, all you see is that your CherryPy server sometimes start spinning.
Besides, the server might run fine for days, and start spinning all the sudden, making it very hard to reproduce it
in your development environment and thus almost impossible to find out where it comes from !


The next section explain how to easily debug that to track down the culprit ...

\section{Debugging a spinning server}
The following method only works on Unix because it uses the \var{gdb} debugger.

All you have to do is wait for your CherryPy server to start spinning. Once it happens, fire up \var{gdb} using the
name of the Python program that's running your CherryPy server (with the correct version). This might be for instance:
\begin{verbatim}
gdb python2.1
or
gdb python2.2
\end{verbatim}
Once you're in \var{gdb}, just attach to the CherryPy process (you can get the process number with the \var{ps} command):
\begin{verbatim}
attach 7457
\end{verbatim}
At this point, \var{gdb} will display a bunch of messages saying that it's reading and loading some symbols.

Now comes the clever trick: run the following command in gdb:
\begin{verbatim}
call PyRun_SimpleString("import sys, traceback; sys.stderr=open('/tmp/tb','w',0); traceback.print_stack()")
\end{verbatim}

This will save the traceback in the \file{/tmp/tb} file. Just exit gdb and look at this file ... It should be obvious where
the server was stuck. In our example, the file contained the following lines:
\begin{verbatim}
  File "TestServer.py", line 454, in ?
    try: _serveForever(_masterSocket)
  File "TestServer.py", line 215, in _serveForever
    _handleRequest(_wfile)
  File "TestServer.py", line 363, in _handleRequest
    response.body=eval("%s.%s(%s)"%(_myClass,_function, _paramStr))
  File "<string>", line 0, in ?
  File "TestServer.py", line 61, in index
    _page.append(str(self.extractText(code)))
  File "TestServer.py", line 55, in extractText
    if j!=-1: code=code[:i]+' '+code[j+1:]
  File "<string>", line 1, in ?
\end{verbatim}

PS: Thanks to Barry Warsaw for this trick (which was originally posted to a Zope mailing list).

\chapter{How to create an XML-RPC server with CherryPy}
Writing an XML-RPC service with CherryPy is very easy. It works very much the same way as a regular web service.
It is base on the \var{xmlrpclib} module. This module is included with Python-2.2 and higher, but if you use a lower
version, you'll have to download it from \url{http://www.pythonware.com}.

All you have to do is add the keyword xmlrpc after the definition of the
mask or view, and before the colon. Now, that view or mask will be
available to XML-RPC requests. 

\section{Basic Example}
Let's take an example. Just create a file called testXmlRpc.cpy
containing the following code: 

\begin{verbatim}
CherryClass Root:
view:
    def add(self, a, b, c) xmlrpc:
        # Return the sum of three numbers
        return a+b+c

CherryClass Xml_rpc:
view:
    def reverse(self, label) xmlrpc:
        # Reverse the characters of a string
        newStr=''
        for i in range(len(label)-1,-1,-1):
            newStr+=label[i]
        return newStr
\end{verbatim}

Also, create a configuration file (textXmlRpcServer.cfg) to tell the
server to accept XML-RPC requests: 

\begin{verbatim}
[server]
typeOfRequests=xmlRpc,web
\end{verbatim}

Compile the source file and start the server: you have an XML-RPC server
running. 

Now let's write a client. Just create a file called testXmlRpcClient.py
containing the following code: 

\begin{verbatim}
import xmlrpclib
testsvr = xmlrpclib.Server("http://localhost:8000")
print testsvr.add(1,2,3)
print testsvr.xml.rpc.reverse("I love cherry pie")
\end{verbatim}

Run the client and you should see: 

\begin{verbatim}
[remi@serveur]$ python xmlrpcTestClient.py
6
eip yrrehc evol I
[remi@serveur]$
\end{verbatim}

\section{Multiple Servers}

CherryPy lets you define your XML-RPC server URIs any way you want. In
our example client above, we used:

\begin{verbatim}
testsvr = xmlrpclib.Server("http://localhost:8000")
print testsvr.xml.rpc.reverse("I love cherry pie")
\end{verbatim}

However, either of the following would work just as well.

\begin{verbatim}
testsvr = xmlrpclib.Server("http://localhost:8000/xml")
print testsvr.rpc.reverse("I love cherry pie")
\end{verbatim}

or

\begin{verbatim}
testsvr = xmlrpclib.Server("http://localhost:8000/xml/rpc")
print testsvr.reverse("I love cherry pie")
\end{verbatim}

Note that in these two examples, the \var{add} view in the \var{Root} CherryClass
would not be available to the XML-RPC client. This gives you the
flexibility to define your XML-RPC APIs to make different methods
available at different URIs.

PS: Note that you can also declare a whole CherryClass as xmlrpc. This is equivalent to declaring each mask and view of this
CherryClass as xmlrpc. For instance, the basic example of this HowTo could have been written as:

\begin{verbatim}
CherryClass Root xmlrpc:
view:
    def add(self, a, b, c):
        # Return the sum of three numbers
        return a+b+c

CherryClass Xml_rpc xmlrpc:
view:
    def reverse(self, label):
        # Reverse the characters of a string
        newStr=''
        for i in range(len(label)-1,-1,-1):
            newStr+=label[i]
        return newStr
\end{verbatim}

\chapter{How to make hidden masks or views}
\section{Introduction}
Masks and views usually correspond to URLs. For instance, if one writes the following code:
\begin{verbatim}
CherryClass CommonMasks:
mask:
    def redLabel(self, label):
        <b><font color=red py-eval="label"></font></b>
CherryClass Root:
mask:
    def index(self):
        <html><body>
            Hello, <py-eval="commonMasks.redLabel('world')">
        </body></html>
\end{verbatim}
The URL \url{http://localhost:8000} will correspond to \var{root.index}.

That's fine, but this also means that if someone artificially types the URL
\url{http://localhost:8000/commonMasks/redLabel?label=IHateCherryPy}, they will get the result of the \var{redLabel} mask.

In this case, it's not a very big deal because the \var{redLabel} mask doesn't do anything important, but in some cases
this might be a problem. That's why \strong{hidden} masks and views were included in CherryPy-0.8.

\section{How it works}
All you have to do is add the keyword \strong{hidden} after the definition of the mask or view, and before the colon.
In our example, one could write:
\begin{verbatim}
CherryClass CommonMasks:
mask:
    def redLabel(self, label) hidden:
        <b><font color=red py-eval="label"></font></b>
CherryClass Root:
mask:
    def index(self):
        <html><body>
            Hello, <py-eval="commonMasks.redLabel('world')">
        </body></html>
\end{verbatim}

All this means is that the \var{redLabel} mask can no longer be accessed directly from the browser. But it can be called
from another mask or view.

It is also possible to declare that an entire CherryClass is hidden, like this:
\begin{verbatim}
CherryClass CommonMasks hidden:
mask:
    def redLabel(self, label):
        <b><font color=red py-eval="label"></font></b>
CherryClass Root:
mask:
    def index(self):
        <html><body>
            Hello, <py-eval="commonMasks.redLabel('world')">
        </body></html>
\end{verbatim}
In this case, all masks and views of \var{CommonMasks} will be hidden.

\chapter{How to control logs}
By default, a CherryPy server outputs all logs to the console. But you can control that by using two configuration variables
in the \strong{[server]} section of the configuration file:
\begin{itemize}
\item
\strong{logToScreen}: This indicates if the logs should be sent to the console or not. Possible values are 0 or 1, default value is 1. Example:
\begin{verbatim}
[server]
socketPort=80
logToScreen=0
\end{verbatim}
\item
\strong{logFile}: Here you can specify the path of a file where all logs will be stored. Example:
\begin{verbatim}
[server]
socketPort=80
logToScreen=0
logFile=/mydir/myLog.log
\end{verbatim}
\end{itemize}

If these options are not enough for you (for instance, you might be running CherryPy on some embedded device with weird
logging constraints), you can define your own special function called \var{logMessage}, which takes two arguments (including the optional level). The default behaviour of this function is to output the message to screen and/or to a file according to
the \var{logToScreen} and \var{logFile} configuration variables.

The following code is an example of how to write your own \var{logMessage} function:
\begin{verbatim}
def logMessage(message, level=0):
    # Only print message if level < 5
    if level < 5:
        print "Here is your message:", message

CherryClass Root:
mask:
    def index(self):
        <html><body>Hello</body></html>
\end{verbatim}

Note that the first line \strong{def logMessage(message, level=0):} has to be exactly that !

\chapter{How to use sessions}
\section{Introduction to sessions}
The main difference between a web application and a regular application is that in a web application, each page requested by
a client is independant from the other pages requested by the same client. In other words, we have to start from
scratch for each page.

Of course, for any serious application, this is not acceptable and we need to keep some data about a given user across
several pages.

Let's assume for example that we want to keep the first name and the last name of the user across several pages. We ask the
user for his first name and last name on the first page of the website (using a form) and we need to use that data in other
pages of the site.

First of all, we have basically two options:
\begin{itemize}
\item
keep the first name and the last name from one page to the next on the client side
\item
store the first name and the last name somewhere on the server side, and only keep
a pointer to that data on the client side: \strong{this is what sessions do}
\end{itemize}

If we choose the first option, we have several ways to do this:
\begin{itemize}
\item
    Keep the first name and the first name in the URL: each time there is a link to another page, we could add the
    following arguments to the URL: "firstName=....\&lastName=..."
\item
Keep the data in hidden fields of a form: in each page, we could have a form with two hidden fields like this:
\begin{verbatim}
<form method="post" name="myForm" action="dummy">
    <hidden name="firstName" value="...">
    <hidden name="lastName" value="...">
</form>
\end{verbatim}
Everytime there is a link to another page, we have to submit the form to get the data. We can use something like this:
\begin{verbatim}
<a href="#"
   onClick="
      document.myForm.action='...put link here...';
      document.myForm.submit();
      return false;">...</a>
\end{verbatim}
\item
Store the first name and the last name in a cookie
\end{itemize}

The first two options are not really handy as they force you to use extra code for each link. The third option is better but
it is not very handy if we have lots of data to keep about a user.

So another option is to store the data on the server side and to just keep a pointer to that data on the client side. This
is what sessions do ...

\section{Possible implementations for sessions}
The pointer to the data is usually called a \strong{sessionId}. Since we need to keep the sessionId from one page
to the next on the client side, we still have the same options as described in the previous section:

\begin{itemize}
\item
Store the sessionId in the URL (this is why you sometimes see long URLs for some sites like \url{http://domain/page?sessionId=dsqlkjsqlkdsklsjsk7987987987987dqkshsqjkhsq798798})
\item
Store the sessionId in the hidden field of a form
future.
\item
Store the sessionId in a cookie
\end{itemize}


The session data itself (in our example: the first name and the last name) is stored on the server side.
Again, there are many options to store it:
\begin{itemize}
\item
Store it in RAM
    \begin{itemize}
    \item
    Advantage: very fast; can store any python object
    \item
    Disavantage: doesn't work in in a multi-process environment; we lose the data when the server is stopped
    \end{itemize}
\item
Store it in the filesystem
    \begin{itemize}
    \item
    Advantage: never lose data; works in a multi-process environment
    \item
    Disavantage: slow; lots of reads/writes to disk; can only store pickable python objects
    \end{itemize}
\item
Store it in a cookie in the client's browser
    \begin{itemize}
    \item
    Advantage: works in multi-thread or multi-process environments. Server is not vulnerable to attacks that consist in artificially starting thousands of sessions on the server to bring it to its knees.
    \item
    Disavantage: can only store pickable python objects. Session data \strong{must} be quite small (some browsers limit the size of cookie data to 4KB)
    \end{itemize}
\item
Store it in a database
    \begin{itemize}
    \item
    Advantage: never lose data; works in a multi-process environment
    \item
    Disavantage: slower than RAM; lots of reads/writes to database; can only store pickable python objects
    \end{itemize}
\item
Store it anywhere you can think of ...
\end{itemize}

Some systems can also mix some of these options: for instance, session data can be stored in RAM and saved to disk or 
to a database once in a while.

Another option is to store everything in RAM and save it to disk when the server shutsdown (although this might be dangerous
because we might not have time to save it if the server crashes or is killed badly).


Also, session data might not change too often so another option is to store it in RAM and save it to disk or to a
database only when it changes.

\section{Sessions implementation in CherryPy}
The following options are supported in CherryPy:

\begin{itemize}
\item
sessionId: for now, the sessionId is always stored in a cookie
\item
session data: three options are build in and you can implement very easily any storage method you want ...
    \begin{itemize}
    \item
    Store everything in RAM; never save it to disk or to a database
    \item
    Store everything in the filesystem; read and save the data for each request
    \item
    Store the session data in the client browser, as a cookie
    \item
    Write your own storage functions so you can store the data wherever you want
    \end{itemize}
\end{itemize}
You should use the first option (RAM) if you don't use multiple processes (using multiple threads is fine) and if you don't care about losing session data when the server is stopped/restarted.

You should use the second option (filesystem) if you don't have a database and don't want to lose data.

The third option is quite interesting but only works if the session data for each user is quite small (some browsers limit the size of cookies to 4KB).

The fourth option (storing your session data in a database) is the recommended option for most "real-world" websites.

\section{Configuration variables used to control sessions}
In order to use sessions, you must first enable sessions by setting a few configuration variables on the config file, under
the \var{[session]} scope.
The new configuration options are:
\begin{itemize}
\item
\var{storageType}: Can be either "ram", "file", "cookie" or "custom": this tells whether you want to store session data in RAM, to disk, in a cookie, or write your own storage functions.
\item
\var{storageFileDir}: This must be set if you set \var{storageType} to "file". Set it to the directory where the session
data will be stored
\item
\var{timeout}: Number of minutes after which a session expires if there was no activity. The default is 60 minutes.
\item
\var{cleanUpDelay}: If cleanUpDelay is set to N minutes, then the CherryPy server will check every N minutes if there are old/expired sessions that need to be cleaned up. The default is 60 minutes.
\item
\var{cookieName}: Name of the cookie that stores the sessionId: The default is "CherryPySession"
\end{itemize}

\section{Cleaning up old sessions}
If \var{storageType} is either "ram", "file" or "cookie", then CherryPy will automatically clean up expired sessions for you. If \var{storageType} is set to "custom", then you have to write your own code to clean up old sessions in a special function called \var{cleanUpOldSessions}.

\section{Using sessions in your code}
Once you have enabled sessions in the config file, the way it works is very easy: CherryPy just makes available for you
a dictionary that can be accessed through \var{request.sessionMap}. You can use this dictionary to store the data that you want to keep about a perticular client.

The reason why \var{sessionMap} is a member variable of \var{request} is because this makes it thread-safe.

\section{Example}
For instance, if you want to store session data in RAM, if you want sessions to expire after 2 hours, and if you want CherryPy to clean up expired
sessions every hour, use the following config file (\file{RootServer.cfg}):
\begin{verbatim}
[session]
storageType=ram
timeout=120
cleanUpDelay=60
\end{verbatim}
If you want to store session data to disk (in a directory called \file{/home/user/sessionData}), use the following
config file:
\begin{verbatim}
[session]
storageType=file
storageFileDir=/home/user/sessionData
\end{verbatim}

The following example is a trivial page view counter:
\begin{verbatim}
CherryClass Root:
mask:
    def index(self):
        <py-code="
            count=sessionMap.get('pageViews', 0)+1
            sessionMap['pageViews']=count
        ">
        <html><body>
            Hello, you've been here <py-eval="count"> time(s)
        </body></html>
\end{verbatim}

\section{Storing session data in a database (or anywhere else)}
From CherryPy-0.9-gamma and later, you can now write your own custom functions to save/load session data. This allows you to store them wherever you want.
In order to do so, all you have to do is declare the three following special functions somewhere in your code:
\begin{itemize}
\item
\var{saveSessionData(sessionId, sessionData, expirationTime)}: Store the sessionData and expiration time for this sessionId somewhere. \var{expirationTime} is a \var{float} as returned by the \var{time.time()} function.
\item
\var{loadSessionData(sessionId)}: Retrieve the sessionData and expiration time for this sessionId and returns them as a tuple
\item
\var{cleanUpOldSessions()}: Delete expired sessions
\end{itemize}

The following example shows how to store the session data in a MySql database. It assumes that there is a table called "session_data":
\begin{verbatim}
mysql> create table session_data(sessionId varchar(50), sessionData text, expirationTime int unsigned);
Query OK, 0 rows affected (0.00 sec)

mysql> describe session_data;
+----------------+------------------+------+-----+---------+-------+
| Field          | Type             | Null | Key | Default | Extra |
+----------------+------------------+------+-----+---------+-------+
| sessionId      | varchar(50)      | YES  |     | NULL    |       |
| sessionData    | text             | YES  |     | NULL    |       |
| expirationTime | int(10) unsigned | YES  |     | NULL    |       |
+----------------+------------------+------+-----+---------+-------+
3 rows in set (0.01 sec)

\end{verbatim}
Note that we've chosen to use an \var{int} column to store the expiration time, which is easier because that's how this variable is passed to the \var{saveSessionData} function. We could have chosen to use a \var{datetime} column but then we would have to convert \var{expiration} time into a suitable format to insert it into the table (and convert it back in \var{loadSessionData}).


Here is an example code that uses the \var{session_data} MySql table to store session data:
\begin{verbatim}
use MySql
import pickle, base64, StringIO, time
CherryClass MyDb(MySql):
function:
    def __init__(self):
        self.openConnection('localhost', 'test', '', 'test')

def saveSessionData(sessionId, sessionData, expirationTime):
    # Pickle sessionData and base64 encode it
    f = StringIO.StringIO()
    pickle.dump(sessionData, f, 1)
    dumpStr = f.getvalue()
    f.close()
    dumpStr = base64.encodestring(dumpStr)
    myDb.query("delete from session_data where sessionId='%s'" % sessionId)
    myDb.query("insert into session_data values('%s', '%s', %s)" % (
        sessionId, dumpStr, int(expirationTime)))

def loadSessionData(sessionId):
    sessionList = myDb.query("select sessionData, expirationTime from session_data where sessionId='%s'" % sessionId)
    if not sessionList:
        return None
    dumpStr = base64.decodestring(sessionList[0][0])
    f = StringIO.StringIO(dumpStr)
    sessionData = pickle.load(f)
    f.close()
    return (sessionData, sessionList[0][1])

def cleanUpOldSessions():
    now = time.time()
    myDb.query("delete from session_data where expirationTime < %s" % int(now))

CherryClass Root:
view:
    def index(self):
        i = request.sessionMap.get('counter', 0) + 1
        request.sessionMap['counter'] = i
        return str(i)
\end{verbatim}

\chapter{How to use Psyco with CherryPy}

Psyco is a python module that can improve the speed of some python programs. Psyco can be found here:
\url{http://psyco.sourceforge.net}.

Pysco has little impact on the speed of the CherryPy code itself (the code is already really fast :-) so you should only use it if you think it
can improve the speed of the code that \strong{you} wrote for your site (for instance, if you use some number-crunching functions to build your pages).

Also, some users reported that Psyco introduces big memory leaks in CherryPy web apps. So if you try to use it and notice that your site leaks
memory, you'll know where to look ...

In order to enable Psyco with CherryPy, you should have Psyco installed on your machine and then all you have to do
is to use the "initProgram" special function and to put the following code
in it:
\begin{verbatim}
def initProgram():
    import psyco
    psyco.full()
\end{verbatim}

\chapter{How to use cookies with CherryPy}
CherryPy uses the \var{Cookie} module from python and in particular the \var{SimpleCookie} object type to handle cookies.
More information can be found here: \url{http://www.python.org/doc/current/lib/module-Cookie.html}

\section{Setting cookies}
In order to send a cookie to a browser, you have to use the global variable \var{response.simpleCookie}, which is
a \var{SimpleCookie} object.

The following code shows how to set a cookie in your CherryPy code:
\begin{verbatim}
CherryClass Root:
view:
    def index(self):
        response.simpleCookie['cookieName']='cookieValue'
        response.simpleCookie['cookieName']['path']='/'
        response.simpleCookie['cookieName']['max-age']=3600
        response.simpleCookie['cookieName']['version']=1
        return "<html><body>Hello, I just sent you a cookie</body></html>"
\end{verbatim}

\section{Reading cookies}
Cookies that are sent by a browser are stored in the global variable \var{response.simpleCookie}, which is
a \var{SimpleCookie} object.

The following code shows how to read a cookie in your CherryPy code:
\begin{verbatim}
CherryClass Root:
mask:
    def index(self):
        <html><body>
            Hi, you sent me <py-eval="len(request.simpleCookie)"> cookies.<br>
            Here is a list of cookie names/values:<br>
            <py-for="cookieName in request.simpleCookie.keys()">
                <py-eval="cookieName+': '+request.simpleCookie[cookieName].value"><br>
            </py-for>
        </body></html>

\end{verbatim}

\chapter{How to use Cheetah templates with CherryPy}

Cheetah is a templating language for python. CherryPy comes with its own templating language (CGTL and CHTL), but if you prefer to use Cheetah, that's very easy to do. More about Cheetah can be found at:
\url{http://www.cheetahtemplate.org/}.

Here is a sample code that shows how to do it:
\begin{verbatim}
#################
# File Root.cpy
#################
from Cheetah.Template import Template
CherryClass Root:
view:
    def index(self):
        template = Template(file='index.tmpl')
        template.colors = ['AA', 'BB', 'CC']
        return template.respond()
    
#################
# File index.tmpl
#################
<html>
<head>
<title>Cheetah Experiment</title>
</head>
#attr $colors = []
#if $colors is []
    #set codelist = ['00','33','66','99','CC','FF']
#else 
    #set codelist = $colors
#end if
<body>
<table border=1>
#for $r in $codelist
    #for $g in $codelist
        <tr>
        #for $b in codelist
        #set color = '#%s%s%s'%(r,g,b)
        <td bgColor = $color>&nbsp;&nbsp; $color &nbsp; </td>
        #end for
    #end for
#end for
</table>
</body>
</html>
\end{verbatim}

Note that this example is not optimized because the template file will be read/parsed/rendered each time the page is displayed. It is quite easy to change that code so it only does it once, or everytime the template file is changed.

Many thanks to "ToddB" for contributing this code :-)

\chapter{How to use streaming with CherryPy}

If you need to return a very big page to a browser, then you might want to use streaming, which means that CherryPy
will start sending the page back to the browser before the page is completely built, and the page will
be returned in chunks.

All you need to do to use streaming is to use the \var{response.wfile} variable, which contains the socket used
to send the response back to the browser. You need to write \strong{all} the data on the socket (including the
response header).

You also need to tell CherryPy that you've already sent the response to the browser and therefore it doesn't have to do it.
To do so, you need to set the \var{response.sendResponse} variable to 0.

The following code shows an example of how to use streaming:
\begin{verbatim}
import time
CherryClass Root:
view:
    def index(self):
        response.wfile.write("HTTP/1.1 200\r\n")
        response.wfile.write("Content-Type: text/plain\r\n\r\n")
        response.wfile.write("First line. Sleeping 2 seconds ...\n")
        time.sleep(2)
        response.wfile.write("Second line. Sleeping 2 seconds ...\n")
        time.sleep(2)
        response.wfile.write("Third and last line")
        response.sendResponse = 0
        return "" # The view still needs to return a string
\end{verbatim}

\chapter{Sample deployment configuration for a real-world website}

This HowTo describes a sample deployment configuration that I believe is well adapted for most "real-world" websites.
It is the configuration used by the \url{http://www.cherrypy.org} website itself.

\section{Hardware}
The website runs on the Celeron 1.3Ghz with 512MB RAM and shares the machine with about 30 other sites.
The choice of the hardware depends mostly on your budget and how much traffic you plan to get...

\section{Software environment}
The site runs on Linux RedHat, but I believe it would run just as well on other OSes (including Windows).
It runs on Python-2.3, but I believe it would run just as well on Python-2.2 or Python-2.1. However, Python-2.3 seems
to be a bit faster than older versions.

\section{CherryPy version}
The site usually uses the latest CVS version from CherryPy (currently the 0.9-final) version.
The 0.9 version seems to be much more stable than previous versions so if you are running an older version, I
highly recommend that you upgrade to that version.

\section{CherryPy server configuration}
The site is deployed as a thread-pool server. I believe this is the best option for most websites.
It is configured to run with 10 threads and since each page is really fast to build, the site should be able to support
many (much more than 10) concurrent users...

The number of threads that you should use for your site depends on many criteria, including the number of concurrent
users you plan to have, the average time your pages take to build and the power (CPU and RAM) of the machine that
will run the site.

\section{CherryPy server deployment}
The site is running "exposed" on port 80, which means that there is no other webserver (like Apache) involved.

In order to listen on port 80, the server has to be started by the user "root". However, running the server as "root"
is not very safe, so we use the special function \var{initAfterBind} to switch the user running the server to another
user. The code is very simple:

\begin{verbatim}
def initAfterBind():
    import os
    # We must switch the group first
    os.setegid(500)  # Replace with desired user id.
    os.seteuid(2524) # Idem
\end{verbatim}

Also, since the server runs a pool of threads, we must switch the user for all these threads as well (otherwise,
only the parent thread will be switched). To do so, we just use the \var{initThread} special function, like this:

\begin{verbatim}
def initThread(threadIndex):
    import os
    # We must switch the group first
    os.setegid(500)  # Replace with desired user id.
    os.seteuid(2524) # Idem
\end{verbatim}


\section{Database configuration}
The site uses MySql as a database backend to log the requests that are being made to the server. Therefore, each request
triggers a database write.

Since the MySQLdb driver doesn't work well (from my experience) if we have several threads sharing the same database
connection, each thread is given its own database connection, so it doesn't interfer with other threads.

This is done very easily using the \var{initThread} special function and the \var{request} special variable, like this:

\begin{verbatim}
def initThread(threadIndex):
    time.sleep(threadIndex * 0.2) # Sleep 0.2 seconds between each new database connection to make sure MySql can keep up ...
    request.connection = MySQLdb.connect('host', 'user', 'password', 'name')
\end{verbatim}

This takes advantage of the fact that \var{request} is a special thread-aware class instance (so each thread can safely
set/get member variables without have to worry about other threads)

Then, when we need to run a query, we just use code like this:

\begin{verbatim}
c=request.connection.cursor()
c.execute(query)
res=c.fetchall()
c.close()
\end{verbatim}

And also, the forum on the website runs on top of ZODB and since ZODB doesn't support multiple threads by itself, I had to
install ZEO so each thread also behaves as a ZEO client.

\section{Sessions}
The site also uses sessions (mostly for the forum, and also for one page in the online demo).
Sessions are configured to store the data in RAM. Since the code for managing sessions is thread-safe,
all threads have access to the same session data and everybody is happy :-)

I believe that other storage types for sessions (file, database, cookie) would work just as well.

\section{Results}
As I write these lines, the site has been running with no problem with this new configuration for about 20 days and
I never had to restart it, proving that it is quite stable. Also, the RAM used by the threads has never increased,
proving that there is no memory leaks ...
It is true that the CherryPy.org website is not a "high-traffic" website, but it is still a good sign.

Note that if you're afraid that your CherryPy server might crash while you're sleeping, you can always set up a cronjob
to check that the site is still running and to restart it if it isn't ...

\chapter{How to stream uploaded files directly to disk}

By default, CherryPy handles in the same way all data posted to the server through a form. In all cases, CherryPy
converts that data to a string or a list of strings. This data can be a short string or a big file that's being uploaded.

In most cases, this is very convenient and it works very well. However, if your application requires users to upload very
big files, then converting them to a string and having this string in memory can be a problem ...

In that case, you may want to write these files directly to a file instead of having them in memory.

Here is how this can be done:
\begin{itemize}
\item
Use the \var{initRequestBeforeParse} special function to tell CherryPy not to parse the POST data itself (by setting \var{request.parsePostData} to 0).
\item
Use the \var{FieldStorage} class of the \var{cgi} module to parse the POST data (by reading \var{request.rfile}) and stream the uploaded file to disk.
\end{itemize}

Here is an example code that does this:

\begin{verbatim}
import cgi

def initRequestBeforeParse():
    if request.path == 'postFile':
        request.parsePostData = 0

CherryClass Root:
mask:
    def index(self):
        <html><body>
        <form method=post action=postFile enctype="multipart/form-data">
            Upload a file: <input type=file name=myFile><br>
            <input type=submit>
        </form>
        </body></html>

view:
    def postFile(self):
        # Use cgi.FieldStorage to parse the POST data
        dataDict = cgi.FieldStorage(fp=request.rfile, headers=request.headerMap, environ={'REQUEST_METHOD':'POST'}, keep_blank_values=1)

        value = dataDict['myFile']

        # Value has 2 attributes:
        #    - filename contains the name of the uploaded file
        #    - file is an input stream opened for reading

        f = open('/tmp/myFile', 'wb')
        while 1:
            data = value.file.read(1024 * 8) # Read blocks of 8KB at a time
            if not data: break
            f.write(data)
        f.close()

        return "<html><body>The file has been saved in /tmp/myFile</body></html>"
\end{verbatim}

Note that the file is not streamed directly from the browser to \file{/tmp/myFile}. Instead, it is saved in a \var{tempfile}
by the \var{cgi} module and then streamed from this \var{tempfile} to \file{/tmp/myFile}.

\end{document}

