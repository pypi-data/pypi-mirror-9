\section{Wave runup over sinusoidal ridges}
This scenario simulates a wave flowing up a beach with skewed sinusoidal ridges. Following the initial wave runup, eventually the water elevation should become constant, and the velocities should approach zero. 

\subsection{Results}
Figure~\ref{fig:vel_t1_centroid} shows the centroid velocities during the wave runup. The flow should be concentrating in the channels near the shore, and be free from major spikes.

\begin{figure}
\begin{center}
\includegraphics[width=0.9\textwidth]{vel_t1_centroid.png}
\caption{Velocity during the wave runup. Point color corresponds to the bed elevation}
\label{fig:vel_t1_centroid}
\end{center}
\end{figure}

Figure~\ref{fig:vel_t2_centroid} shows the velocities profile at time 40~s. They should be nearly zero (e.g. O($10^{-3}$) m$/s$). This case has been used to illustrate wet-dry artefacts in some versions of \anuga.

\begin{figure}
\begin{center}
\includegraphics[width=0.9\textwidth]{vel_t2_centroid.png}
\caption{Velocity at time 20s after the wave runup. The flow speed should be nearly zero. Point color corresponds to the bed elevation.}
\label{fig:vel_t2_centroid}
\end{center}
\end{figure}

\endinput
