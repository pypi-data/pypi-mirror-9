
\section{Okushiri Benchmark}

We set up the Okushiri Island benchmark as published by the
\begin{verbatim}
The Third International Workshop on Long-Wave Runup Models
June 17-18 2004
Wrigley Marine Science Center
Catalina Island, California
\end{verbatim}
\url{http://isec.nacse.org/workshop/2004_cornell/}


The validation data was downloaded and made available in this directory
for convenience but the original data is available at
\url{http://isec.nacse.org/workshop/2004_cornell/bmark2.html}
where a detailed description of the problem is also available.


Run \verb|create_okushiri.py| to process the boundary condition and build the
mesh before running the script \verb|run_okushiri.py|.

\subsection{Results}

\anuga{} should produce results that match the stage values at a number of gauge locations



\begin{figure}[h]
\begin{center}
\includegraphics[width=0.9\textwidth]{Boundary.png}
\end{center}
\caption{Stage at boundary gauge}
\label{okushiri:boundary}
\end{figure}



\begin{figure}[h]
\begin{center}
\includegraphics[width=0.9\textwidth]{ch5.png}
\end{center}
\caption{Stage at gauge station 5}
\label{okushiri:ch:five}
\end{figure}

\begin{figure}[h]
\begin{center}
\includegraphics[width=0.9\textwidth]{ch7.png}
\end{center}
\caption{Stage at at gauge station 7}
\label{okushiri:ch_seven}
\end{figure}

\begin{figure}[h]
\begin{center}
\includegraphics[width=0.9\textwidth]{ch9.png}
\end{center}
\caption{Stage at at gauge station 9}
\label{okushiri:ch_nine}
\end{figure}




\endinput