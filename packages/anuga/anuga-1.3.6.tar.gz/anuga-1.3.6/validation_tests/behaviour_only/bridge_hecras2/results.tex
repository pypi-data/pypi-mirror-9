\section{Bridges in HECRAS and \anuga{} using an internal boundary operator}
This test compares a prismatic channel flow with a bridge in HECRAS and ANUGA.
A 10m wide, 1000m long channel (slope of 1/200, bankfull depth of 1m,
rectangular cross-section) flows through a floodplain (10m wide on either side
of the channel) (Figure~\ref{schematic}). 500m downstream there is a bridge
with a 1.3m high rectangular opening over the channel, and a deck elevation of
-1m. In HECRAS the bridge is modelled using the energy method, see the
associated HECRAS files for details. In ANUGA the bridge is modelled by
inserting the bridge deck (upper chord) into the topography, with an internal
boundary operator to describe the bridge underflow. The rating curves for the
bridge underflow (used to compute the bridge discharge from the upstream and
downstream stage) were derived for ANUGA from HECRAS by raising the upper chord
of the bridge in HECRAS (far about the flow) and computing internal boundary
tables, which were then copied into a csv file for use by ANUGA. The bridge
overflow in ANUGA is modelled with the shallow water equations (although
riverwalls could also be used to apply weir type equations instead). Both
models have a uniform Manning's n of 0.045, which prevents too much
supercritical flow in HECRAS (and the associated numerical supression of the
inertial terms that HECRAS uses to retain stability).

\begin{figure}
\begin{center}
\includegraphics[width=1.0\textwidth]{hecras_bridge_test/RASGeometry_Bridge.png}
\end{center}
\caption{Screenshot showing the HECRAS model geometry schematization}
\label{schematic}
\end{figure}


A discharge timeseries is imposed upstream for both models, with the discharge
increasing from 1m$^3$/s initially to 70m$^3$/s at the end of the simulation.
Details of the model setup can be seen in the code / input files in this
directory.

\subsection{Results}

Figure~\ref{Reach} show stage timeseries at various stations downstream in each
model. The ANUGA and HECRAS results are qualitatively similar, but differ in
detail. In early stages of the simulation when discharges are lower, ANUGA
shows stages slightly below HECRAS. This reflects the fact that HECRAS models
side-wall friction while ANUGA does not, so there is more drag in the HECRAS
model. Another relevant factor may be that the models flux the momentum under
the bridge in different ways. ANUGA's method is to compute the average momentum
in each direction along the upstream bridge inflow, and assume that this is
advected by the discharge (as computed from the internal boundary rating
curves). HECRAS's method is based on the energy equation.

As the discharge increases, the models show more deviation around the
bridge and upstream, and ultimately approach different steady states. The main
reason for this is that they use different methods to model the bridge
overflow, which begins when station 525 exceeds -1m. Another cause of
differences is that in HECRAS, stage over all cross-sections is constant,
including just upstream and downstream of the bridge. On the other hand in
these locations ANUGA predicts some cross-channel variations in channel and
floodplain stage. This is caused by the flux of water under the centre of the
bridge. Just upstream of the bridge, ANUGA predicts the stage in the channel is
slightly lower than on the floodplains, whereas the reverse is true just
downstream, as would be expected from the bridge underflow. Combined with the
fact that ANUGA's enquiry points for the bridge underflow rating curves occur
in the centre of the channel, we cannot expect exact agreement with HECRAS.
HECRAS includes several other bridge models, and these would also give
different results particularly when the bridge deck is inundated. 

\begin{figure}
\begin{center}
\includegraphics[width=0.9\textwidth]{CENTRAL_CHANNEL.png}
\end{center}
\caption{Stage at various points downstream in the channel}
\label{Reach}
\end{figure}

