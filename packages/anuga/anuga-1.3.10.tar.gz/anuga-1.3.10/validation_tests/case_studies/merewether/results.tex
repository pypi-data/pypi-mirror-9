\documentclass{article}
\usepackage{datatool}
\usepackage{graphicx}
\usepackage{datatool}

\begin{document}

\title{ANUGA applied to the ARR Project 15 Urban Flood Modelling test case}

\maketitle

\section{Introduction}
This is a model of the Pasha Bulker flood (June 2007) in part of Merewether,
Newcastle, Australia. It was developed by the UNSW Water Reseach Laboratory (WRL),
and various information on the event is reported in `ARR Project 15: Two
dimensional simulations in urban areas, Representation of buildings in 2D
numerical flood models'. This chapter should be consulted for details on this
test case, and reports on the performance of other models. 

The ANUGA model was initially developed by WRL (who gave us permission to
include it here) and further refined by ANUGA developers to include variable
friction (with roads having a manning's n of 0.02, and everywhere else having
0.04), an inlet operator, and options for either using buildings as topography, or
buildings cut out of the mesh.

\section{Results}
Table~\ref{tab:comparison} compares the observed peak stage during the flood to
ANUGA, and also values for a TUFLOW model which are reported in the ARR Project
15 report. The ANUGA values should be similar to the field observations, and
even more-so to the TUFLOW numerical model values. This suggests both models
are giving a similar solution, with the main source of errors being related to
the input data and the shallow water assumptions, rather than the numerical
scheme.

\DTLloaddb{comparison}{Stage_point_comparison.csv}
\begin{table}
\caption{Comparison of peak stage field observations, the ANUGA model, and a
TUFLOW model (developed by WRL). See the ARR project 15 report for more
information}
\label{tab:comparison}
\DTLdisplaydb{comparison}
\end{table}

Figure~\ref{fig:stationary_vel} shows a vector plot of the velocities around some buildings. It should
show a similar pattern to Figure 30 in the ARR Project 15 report.
\begin{figure}
\includegraphics[width=\textwidth]{velocity_stationary.png}
\caption{Velocity vectors around some buildngs at the end of the simulation
(where it should have reached stationary state). Compare with Figure 30 in the ARR Project 15 report (the results should be similar)}
\label{fig:stationary_vel}
\end{figure}

Figure~\ref{fig:transect1} shows a transect of velocity and depth between 2 buildings (Transect
1 in the ARR Report).  It should show a similar pattern to the TUFLOW and
MIKE21 models described in the ARR Report (their Figures 27 and 28).  In this
site the numerical models show some differences to the physical model data,
which may reflect the limitations both of the physical model, and of the
shallow water assumptions (see ARR Report for further discussion).
\begin{figure}
\includegraphics[width=\textwidth]{Transect1.png}
\caption{Velocity and depth along Transect 1. Compare with Figures 27 and 28 
in the ARR Project 15 report (the results should be similar to TUFLOW and
MIKE21)}
\label{fig:transect1}
\end{figure}

Figure~\ref{fig:froude} shows the froude number (as either 0, $(0-1]$ or $>1$).  It
should show a similar pattern to Figure 29 in the ARR report, although the
colors are different. 
\begin{figure}
\center
\includegraphics[width=0.7\textwidth]{froudeNumber.png}
\caption{Froude number (0, $(0-1]$, or $>1$). The largest category area is
zero, the smallest category area is $>1$. Compare with Figure 29 in ARR report
(the patterns should be quite similar).}
\label{fig:froude}
\end{figure}

\end{document}
