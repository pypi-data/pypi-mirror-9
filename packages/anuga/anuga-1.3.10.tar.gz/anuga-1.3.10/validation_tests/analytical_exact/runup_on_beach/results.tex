\section{Simple wave runup}
This scenario simulates a wave flowing up a planar beach. Following the initial wave runup, eventually the water elevation should become constant, and the velocities should approach zero. This test follows from the \anuga{} User Manual~\cite{RNGS2010}.
Instead of demonstrating how \anuga{} simulates a water flow test as given in the \anuga{} User Manual~\cite{RNGS2010}, here we investigate the behaviour of the wetting (and possibly the drying) process handled by \anuga{}.

\subsection{Results}
To do the investigation, we consider initial and boundary conditions different from those given in the \anuga{} User Manual~\cite{RNGS2010} so that we have a large dry region. Consider a rectangular domain with $x\in[0,1]$ and $y\in[0,0.03]$\,. The initial conditions are $u=v=0$ and 
\begin{equation}
z(x) = -\frac{x}{2}
\end{equation}
with the stage at wet region is $w=-0.45$. The boundary condition on the rightend of the domain is Dirichlet with stage $w=-0.4$ and discharges in $x$ and $y$ directions are zero.

At an early runup, representatives of the results are as follows. Figure~\ref{fig:stage_1s} shows the water surface at time $t=1$ (in the cross-shore direction). It is not constant as the water is flowing up the beach at this time. Figure~\ref{fig:xvel_1s} shows the corresponding $x$-velocity during the wave runup. The velocities should be free from major spikes.
\begin{figure}
\begin{center}
\includegraphics[width=0.9\textwidth]{stage_1s.png}
\caption{Water surface during the wave runup at time $t=1.0$\,.}
\label{fig:stage_1s}
\end{center}
\end{figure}

\begin{figure}
\begin{center}
\includegraphics[width=0.9\textwidth]{xvel_1s.png}
\caption{Xvelocity during the wave runup at time $t=1.0$\,.}
\label{fig:xvel_1s}
\end{center}
\end{figure}

After a much longer time, representatives of the results are as follows. Figure~\ref{fig:stage_30s} shows the water surface at time 30s (in the cross-shore direction). It should be nearly constant (= -0.1m) in the wet portions of the domain. Figure~\ref{fig:xvel_30s} shows the corresponding velocity at time 30s. It should be nearly zero (e.g. $<<$ 1 mm/s). This case has been used to illustrate wet-dry artefacts in some versions of \anuga.
\begin{figure}
\begin{center}
\includegraphics[width=0.9\textwidth]{stage_30s.png}
\caption{Water surface at time 30s after the wave runup. It should be nearly constant in wet parts of the domain.}
\label{fig:stage_30s}
\end{center}
\end{figure}

\begin{figure}
\begin{center}
\includegraphics[width=0.9\textwidth]{xvel_30s.png}
\caption{Xvelocity at time 30s after the wave runup. It should be nearly zero.}
\label{fig:xvel_30s}
\end{center}
\end{figure}

\subsection{Comment on wet-dry artefacts}
Earlier \anuga{} algorithms showed some wet-dry artefacts in this scenario, which manifest as a high velocity at the wet/dry interface, and water creeping up the slope (this was a feature of earlier \anuga{} algorithms). If the process is handled correctly in \anuga{}, water should not ``creep-up`` to the left for a long distance (the discontinuous elevation + tsunami algorithms should have no problem with this).

%For one of the solvers with these artefacts, setting the minimum allowed height and the minimum storable height in an appropriate way is important. If the minimum allowed height is different from the minimum storable height, water may ``creep-up.`` That can be demonstrated by taking, for example, the minimum allowed height is $10^{-3}$ and the minimum allowed height is $10^{-5}$. However if we take those two parameters to be $10^{-4}$, water is still after we run the code for a relatively long time. The following lines can be implemented in the code:
%\begin{verbatim}
%domain.set_minimum_allowed_height(0.0001)
%domain.set_minimum_storable_height(0.0001)
%\end{verbatim}

\endinput
